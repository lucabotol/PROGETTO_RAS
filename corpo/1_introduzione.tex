\section{Introduzione}
L'aumento della frequenza di eventi estremi pluviometrici, causati dal cambiamento climatico, porta con sè la necessità per gli amministratori del territorio a comprendere maggiormente le risposte idrauliche delle aree a maggior rischio.\\
Per tale motivo, alle formule teoriche di studio dei deflussi vengono affiancati i modelli digitali di analisi idraulica. In tali strumenti, a fini di studio, possono essere introdotti valori fisici reali oppure previsionali degli eventi atmosferici.\\ 
In questa relazione si andrà ad esporre i procedimenti necessari per svolgere l'analisi di risposta idraulica di due bacini montani, mediante l'analisi statistico-probabilistica delle piogge e l'utilizzo di un software di modellazione idrologica (HEC-HMS).\\
La prima parte della relazione interesserà l'analisi statistico-probabilistica delle serie storiche pluviometriche, in modo da ottenere la corretta curva LSPP per un dato tempo di ritorno (215 anni).\\
Successivamente, utilizzando i dati di pioggia e di deflusso misurati durante l'evento Vaia, sarà possibile creare ed ottimizzare il modello idrologico dei due bacini di studio.\\
Infine, dal risultato dell'analisi statistico-probabilistica, si riuscirà ad ottenere le risposte di deflusso dei bacini, a fronte di una qualsiasi precipitazione con un dato tempo di ritorno.\\
Essendo una relazione tecnica, si andrà ad esporre ed elencare solamente le procedure svolte per l'analisi idrografica, considerando che il lettore abbia una conoscenza basilare di utilizzo del programma HEC-HMS.