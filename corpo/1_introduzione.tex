\section{Introduzione}
L'aumento della frequenza di eventi estremi pluviometrici, causati dal cambiamento climatico, porta con sè la necessità per gli amministratori del territorio a comprendere maggiormente le risposte idrauliche delle aree a maggior rischio.\\
Per tale motivo, alle formule teoriche di studio dei deflussi vengono affiancati i modelli digitali di analisi idraulica. In tali strumenti, a fini di studio, possono essere introdotti valori fisici reali oppure previsionali degli eventi atmosferici.\\ 
In questa relazione si andrà ad esporre i procedimenti necessari per svolgere l'analisi di risposta idraulica di una zona di confluenza di due bacini montani, mediante l'utilizzo di un software di modellazione idraulica (HEC-RAS).\\
La prima parte della relazione interesserà la creazione dell'area dove si andrà a simulare l'evento idraulico.\\
Essendo una relazione tecnica, si andrà ad esporre ed elencare solamente le procedure svolte per l'analisi idraulica, considerando che il lettore abbia una conoscenza basilare di utilizzo del programma HEC-RAS.