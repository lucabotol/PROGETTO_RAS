\section{Normative e teoria sul rischio idraulico}
Per quanto riguarda la materia legislativa inerente al rischio idraulico, è necessario sapere che esistono diverse norme (regionali, nazionali o comunitarie) che afferiscono alla stessa materia.\\
La norma più importante è la Direttiva Europea 2007/60/CE (anche detta \textit{floods directive}), che è stata attuata in Italia con il Decreto Legislativo 49/2010; tale legge pone le basi per la corretta valutazione e gestione del territorio, per quanto riguarda il rischio di alluvioni.\\
L'obiettivo finale della Direttiva Alluvioni è quello di ridurre gli effetti negativi sulla salute umana, sull'ambiente o sulle infrastrutture.\\
Per fare ciò, la norma prevede un preciso percorso, guidato da specifici obblighi, affinché tutti i territori ricadano all'interno di un ``Piano di Gestione del Rischio Alluvione" (PGRA).\\
Il PGRA viene definito dall'Autorità di Bacino Distrettuale, ed aggiornato con cadenza di 6 anni.\\
Il Triveneto (Province Autonome di Trento e Bolzano, Veneto e Friuli Venezia Giulia), ricade all'interno dell'Autorità di Bacino del Distretto delle Alpi Orientali \cite{distrettoalpiorientali}.\\
Il PGRA attualmente vigente per il Distretto delle Alpi Orientali è stato approvato nel 2021 ed avrà validità fino al 2027 \cite{pgra}.\\
Con rischio idraulico s'intende la combinazione tra la probabilità di avvenimento di un evento alluvionale e la probabilità di generare conseguenze negative per l'ambiente, per l'uomo o per le opere presenti nel territorio. Quindi, il rischio (R) viene ricavato dalla formula:
\begin{equation}
    R = P \cdot V \cdot E
\end{equation}
Dove: 
\begin{itemize}
    \item P: pericolosità, ovvero la probabilità di avvenimento di un certo evento, fissato un tempo di ritorno;
    \item V: vulnerabilità, stima del grado di danneggiamento che un'opera può subire, a fronte di un qualsiasi evento di piena;
    \item E: esposizione, numero di elementi/unità esposti al pericolo in una data area.
\end{itemize}

Invece, con il termine pericolosità s'intende una valutazione (quantitativa) del grado di pericolo, mediante la combinazione tra l'intensità del fenomeno e la sua probabile frequenza di avvenimento. Infatti, per ogni combinazione tra i due fattori è possibile attribuire ad una certa area un preciso valore di pericolo.\\
L'attuale PGRA, per le il Distretto delle Alpi Orientali, prevede tre livelli di pericolo (P1, P2 e P3), che vanno ad indicare (in senso crescente) il livello di rischio idraulico. Tali suddivisioni sono riportate in modo preciso all'interno delle mappe di pericolosità.\\
La Direttiva Alluvioni (ovvero la massima norma idraulica europea) non stabilisce i criteri con cui differenziare le frequenze di accadimento degli eventi, perciò tali valori vengono stabiliti dalle Autorità sottostanti alla UE.\\
Il Distretto delle Alpi Orientali discretizza gli eventi di deflusso come:
\begin{itemize}
    \item alluvioni estreme: tempo di ritorno di 300 anni;
    \item alluvioni poco frequenti: tempo di ritorno di 100 anni;
    \item alluvioni frequenti: tempo di ritorno di 30 anni.
\end{itemize}
Inoltre, il Distretto delle Alpi Orientali pone dei criteri (di velocità o profondità) per differenziare le classi d'intensità degli eventi:
\begin{itemize}
    \item intensità bassa: profondità inferiore al metro;
    \item intensità media: profondità superiore o uguale al metro;
    \item intensità elevata: velocità superiore al metro al secondo.
\end{itemize}

Il Distretto delle Alpi Orientali non pone vincoli sui metodi di simulazione idraulica da presentare, bensì suggerisce l'utilizzo dei principali programmi disponibili sul mercato (HEC-RAS, BASEMENT o il DFRM) e pone dei requisiti minimi di risoluzione della cella di mesh (non superiori ai 10 metri di lato).\\
Il Distretto delle Alpi Orientali mette a disposizione, nel suo sito internet, di diverse risorse digitali, come per esempio cartografie digitali del terreno o specifici layer GIS. 
