\section{Procedimento su HEC-HMS}
Dopo aver ottenuto la funzione della LSPP per il proprio tempo di ritorno, è possibile proseguire con l'analisi idrologica su HEC-HMS.\\
Per fare ciò, è necessario creare un modello idrologico per ogni bacino, in modo da immettere in input i valori pluviometrici calcolati precedentemente.\\
Il \textit{modello idrologico} è una semplificazione del fenomeno reale (che nel nostro caso è la trasformazione degli afflussi in deflussi).\\
Il modello idrologico, affinché possa essere considerato nella simulazione, deve avere i seguenti parametri:
\begin{itemize}
    \item bacino idrologico: indica la struttura del bacino, comprendente gli eventuali sottobacini o particolari elementi idraulici (come per esempio i canali fluviali o serbatoi);
    \item modello meteorologico: sintetizza le caratteristiche pluviometriche dell'evento meteorologico, come per esempio i bacini interessati o gli effetti ambientali del vento e dell'evapotraspirazione;
    \item specifiche di controllo: contiene le caratteristiche generali della simulazione da effettuare, come per esempio il momento di inizio e di fine;
    \item serie di dati temporali: rappresentano i valori (noti) che è possibile introdurre nel modello meteorologico, come per esempio valori pluviometrici o di deflusso. Come riportato successivamente, in questa finestra sono stati introdotti i valori pluviometrici e di deflusso reali dell'evento Vaia, in modo da calibrare il modello.
\end{itemize}
 Ulteriori proprietà, come per esempio i modelli digitali del terreno, possono essere introdotti successivamente nel software.\\
 Come anticipato precedentemente, in questa relazione si andrà a studiare la risposta idraulica di due bacini, quindi risulta necessario creare due modelli di calcolo.