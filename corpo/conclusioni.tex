\section{Conclusioni}
Dopo aver analizzato i risultati delle simulazioni, è possibile trarre alcune conclusioni.\\
Per esempio, è possibile analizzare la questione inerente al tempo computazionale: se le simulazioni fossero state fatte con un valore temporale inferiore, i risultati ottenuti sarebbero qualitativamente migliori, poiché ci sarebbero maggiori operazioni di calcolo svolte dal software.\\
Un maggior raffittimento della mesh, ed ulteriori breaklines, renderebbero il risultato finale più realistico, perché si andrebbe ad evidenziare maggiormente le caratteristiche del suolo (attraverso il modello digitale del terreno).\\
I diversi risultati ottenuti per ogni singola simulazione, come per esempio la portata esondata o l'altezza del tirante, sono coerenti tra di loro; ovvero, il comportamento del corso d'acqua, a seconda del punto in cui viene osservato, ha caratteristiche comuni.\\
Analizzando il risultato finale delle tre simulazioni, è necessario esporre una discordanza rispetto alle aspettative iniziali. Partendo dall'ipotesi iniziale che correlava in modo direttamente proporzionale la scabrezza dell'alveo con l'esondazione dei fiumi, i risultati della prima ed ultima simulazione sono concordi a ciò. La seconda simulazione invece, che dovrebbe generare un'esondazione intermedia alle altre due, crea invece un volume esondato maggiore rispetto alla prima.\\
Evidentemente, al contrario dei due eventi con rugosità estrema, nella simulazione con scabrezza intermedia l'evento di piena genera un'esondazione maggiore verso l'abitato poiché ha nello stesso momento un valore medio di velocità e di altezza del tirante.
